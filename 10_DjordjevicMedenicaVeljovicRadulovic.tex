\documentclass[a4paper]{article}

\usepackage{color}
\usepackage{url}
\usepackage[T2A]{fontenc}
\usepackage[utf8]{inputenc}
\usepackage{graphicx}
\usepackage[english,serbian]{babel} %AKO BUDETE PISALI CIRILICOM ONDA IZMENITE OVU LINIJU KODA
\usepackage[unicode]{hyperref}
\hypersetup{colorlinks,citecolor=pink,filecolor=green,linkcolor=blue,urlcolor=blue} %STAVILA SAM ROZE BOJU ZA CITE AKO VAM SE NE SVIDJA MENJAJTE 
\newtheorem{primer}{Primer}[section]

\begin{document}

\section{Rana istorija kompanije IBM}
Koreni IBM-a naziru se već krajem devetnaestog veka kada su nastale četiri manje kompanije koje su se kasnije spojile u kompaniju IBM. Tehnološki najnapredniju od ove četiri kompanije osnovao je Herman Holerit(eng.~{\em Herman Hollerith}), koji je svojom mašinom pomogao da se popis stanovništva Sjedinjenih Američkih Država 1890. godine uradi efikasnije i brže. Nakon prvog popisa, američka vlada mu je dala ugovor i za naredni popis 1900. godine, ali s obzirom da se popis održava svakih 10 godina, kompanija nije mogla da opstane u periodu između dva popisa. Holerit prodaje svoju kompaniju 1911. godine

\section{Kompanija IBM u sadašnjosti i budućnosti}
Kompanija IBM danas nije isto što i kompanija IBM pre 30 godina. Naime, pre 30 godina IBM bila je jedna od najmoćnijih u svetu tehnologije. Prodavali su svoje PC računare i u tome su bili veoma uspešni.
Danas, IBM više nema licencu za svoje PC računare, jer su ih prodali kineskoj kompaniji \textbf{Lenovo} (OVDE STAVITI CITE ZA LITERATURU). I oni veoma uspešno plivaju u prodaji laptop računara, pre svega serije \textbf{ThinkPad} koja je originalno napravljena od strane IBM-a. Iako je kompanija IBM godinama bila dominantna, danas više teži ka propadanju. Međutim neki analitičari smatraju da je moguć povratak u visoki rang, zajedno sa njihovom dugogodišnjom konkurencijom, jer se trenutno bave cloud servisima i procene su da je moguće da napreduju u periodu između 2022. i 2024. godine.

\begin{primer}
Najveća konkurencija su kompanije Google, Microsoft i Amazon.
\end{primer}

Velika greška ove kompanije je pre svega kasno izbacivanje proizvoda na berzu, do toga je dolazilo iz više razloga. Najviše zbog:

\begin{itemize}
\item detaljnih analiza i testiranja proizvoda
\item nepoznavanja marketa i berze
\item nesuglasica između kolega u kompaniji...
\end{itemize}

IBM se bavi i veštačkom inteligencijom(eng.~{\em Artificial intelligence}). Najpoznatiji projekat, koji ih je i stavio na loš glas u svetu tehnologije, je robot Vatson(eng.~{\em Watson}) za koga je zamisao bila da može da odgovori na sva postavljena pitanja i namena je bila u medicinske svrhe. Međutim, ispostavilo se da robot nije radio ono što su oni zamislili i da je samo mogao da razume reči i da ih čita, i uprkos kritikama od strane stručnih ljudi, IBM odlučuje da Vatson-a izbaci u prodaju. Prodaja nije prošla dobro i to je najveći neuspeh ove kompanije. Odlučuju se da ga prodaju 2022. godine.

Kompanija se 2021. godine deli na dve dela, kao strategija za napredovanje. Odvaja se odeljenje Managed Infrastructure Services, a ostatak kompanije će imati fokus na hibridnu cloud platformu, kao i implementaciju tehnologija koje se tiču veštačke inteligencije.


\begin{table}[h!]
\begin{center}
\caption{Zarada kompanije IBM u prethodnih 5 godina i broj zaposlenih.(OVDE UBACITI CITE ZA LITERATURU)}
\begin{tabular}{|c|c|c|c|c|} \hline
Year& Net income in mil.USD& Total assets in mil. USD& Price per share in USD& Employees \\ \hline
2017	&5,753	&125,356	&149.76	&366,600\\ \hline
2018	&8,723	&123,382	&139.90	&350,600\\ \hline
2019	&9,400	&152,186	&126.85	&352,600\\ \hline
2020	&5,590	&155,971	&125.88	&345,900\\ \hline
2021	&5,743	&132,001	&133.66	&282,100\\ \hline

\end{tabular}
\label{tab:tabela1}
\end{center}
\end{table}


% literatura za sadasnjost: https://www.thecoldwire.com/what-happened-to-ibm/
% literatura za tabelu: https://en.wikipedia.org/wiki/IBM


\end{document}
