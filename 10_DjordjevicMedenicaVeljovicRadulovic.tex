\documentclass[a4paper]{article}
\usepackage{color}
\usepackage{url}
\usepackage[T2A]{fontenc}
\usepackage[utf8]{inputenc}
\usepackage{graphicx}
\usepackage[english,serbian]{babel} %AKO BUDETE PISALI CIRILICOM ONDA IZMENITE OVU LINIJU KODA
\usepackage[unicode]{hyperref}
\hypersetup{colorlinks,citecolor=blue,filecolor=green,linkcolor=blue,urlcolor=blue} 
\newtheorem{primer}{Primer}[section]
\title{Kompanija IBM \\
\normalsize Seminarski rad u okviru kursa\\ Tehničko i naucno pisanje
\\Matematički fakultet}
\author{Staša Đorđević \and
Jovana Medenica \and
Marko Veljović \and
Matija Radulović} %TODO ako treba mejl indeks itd pored imena
\date{15. novembar 2022.} %TODO

\begin{document}
<<<<<<< HEAD
=======
\maketitle
%TODO mozda dodati sazetak 
\tableofcontents
\section{Uvod}
OVDE CU NAPISATI UVOD KAD SVE ZAVRSIMO %TODO
>>>>>>> 51c243093270c5f174028a3e5460c8818e49565c
\section{Rana istorija kompanije IBM}
Koreni IBM-a naziru se već krajem devetnaestog veka kada su nastale četiri manje kompanije koje su se kasnije spojile u kompaniju IBM. Tehnološki najnapredniju od ove četiri kompanije osnovao je Herman Holerit(eng.~{\em Herman Hollerith}), koji je svojom mašinom pomogao da se popis stanovništva Sjedinjenih Američkih Država 1890. godine uradi efikasnije i brže. Nakon prvog popisa, američka vlada mu je dala ugovor i za naredni popis 1900. godine, ali s obzirom da se popis održava svakih 10 godina, kompanija nije mogla da opstane u periodu između dva popisa. Holerit prodaje svoju kompaniju 1911. godine Čarlsu Flintu(eng.~{\em Charles Flint}) za 2,3 miliona dolara koji spaja Holeritovu kompaniju sa ostale tri i tada nastaje kompanija CTR(eng.~{\em Computing-Tabulating-Recording Company}). Flint je 1914. godine u kompaniju doveo Tomasa J. Votsona(eng.~{\em Thomas J. Watson}) koji je iste godine postao generalni menadžer kompanije, a 1915. godine postaje i njen predsednik. Zahvaljujući iskustvu u svom poslu, Votson je ubrzo pri svom dolasku podigao atmosferu u kompaniji poboljšavanjem odnosa sa klijentima i zaposlenima, ali takođe i uvođenjem strogih pravila za zaposlene kao što je potpuna zabrana alkohola. Votson 1915. godine smišlja slogan kompanije, ``MISLI``(eng.~{\em ``THINK``}). Kompanija 1924. godine menja ime u IBM(eng.~{\em International Business Machines}).

\section{IBM sredinom prošlog veka}
\subsection{Kompanija za vreme Velike ekonomske krize(1929-1938)}
Velika ekonomska kriza predstavljala je izazov kakav ranije nije bio viđen u svetu ekonomije, Votson se hrabro suočio sa ovim izazovom i nastavio da investira u zaposlene, proizvodnju i tehnološke inovacije. Umesto da daje otkaz zaposlenima, on je angažovao nove radnike i dao im je beneficije. IBM je jedna od prvih kompanija koja je uvela životno osiguranje i plaćene godišnje odmore za svoje zaposlene. Votson je ovim investicijama doveo kompaniju u ogroman rizik, ali rizik se na kraju isplatio. IBM je iz Velike ekonomske krize izašao u velikom profitu, što će postaviti temelje kompanije za narednih 50 godina.
\subsection{Kompanija za vreme Drugog svetskog rata(1939-1945)}
Pre početka rata, IBM je imao svoje objekte širom sveta, i u zemljama koje su pripadale Saveznicima i u zemljama sila Osovine. Ovi objekti koji su pripadali neprijateljskoj teritoriji(silama Osovine) su uglavnom bivali okupirani od strane lokalne vojske, ali sedište kompanije u Nju Jorku(eng.~{\em New York}) je nastavilo sa radom i za vreme rata pružalo je pomoć SAD-u. Dosadašnja proizvodnja je zamenjena sa proizvodnjom ratne opreme, kao što su računari vojne svrhe i puške. Za vreme ovog perioda napravljen je računar Harvard Mark I, koji je mogao da prima i izvršava programe sa bušenih kartica. Jedan od prvih programa za Harvard Mark I je dizajniran od strane Džon fon Nojmana(eng.~{\em John von Neumann}). Za vreme rata, IBM je proširio svoj kapacitet proizvodnje, dodali su nove zgrade u Nju Jorku, Vašingtonu(eng.~{\em Washington, D.C.}) i San Hoseu(eng.~{\em San Jose}). IBM-ovo prisustvo na zapadnoj obali SAD-a je privuklo i ostale kompanije da tamo prošire svoje ustanove, kasnije će ova oblast oko zaliva San Francisko(eng.~{\em San Francisco}) biti poznata kao Silikonska dolina(eng.~{\em Silicon Valley}).

\section{Kompanija IBM u sadašnjosti i budućnosti}
Kompanija IBM danas nije isto što i kompanija IBM pre 30 godina. Naime, pre 30 godina IBM bila je jedna od najmoćnijih u svetu tehnologije. Prodavali su svoje PC računare i u tome su bili veoma uspešni.
Danas, IBM više nema licencu za svoje PC računare, jer su ih prodali kineskoj kompaniji \textbf{Lenovo}\cite{lit1}. I oni veoma uspešno plivaju u prodaji laptop računara, pre svega serije \textbf{ThinkPad} koja je originalno napravljena od strane IBM-a. Iako je kompanija IBM godinama bila dominantna, danas više teži ka propadanju. Međutim neki analitičari smatraju da je moguć povratak u visoki rang, zajedno sa njihovom dugogodišnjom konkurencijom, jer se trenutno bave cloud servisima i procene su da je moguće da napreduju u periodu između 2022. i 2024. godine.

\begin{primer}
Najveća konkurencija su kompanije Google, Microsoft i Amazon.
\end{primer}

Velika greška ove kompanije je pre svega kasno izbacivanje proizvoda na berzu, do toga je dolazilo iz više razloga. Najviše zbog:

\begin{itemize}
\item detaljnih analiza i testiranja proizvoda
\item nepoznavanja marketa i berze
\item nesuglasica između kolega u kompaniji...
\end{itemize}

IBM se bavi i veštačkom inteligencijom(eng.~{\em Artificial intelligence}). Najpoznatiji projekat, koji ih je i stavio na loš glas u svetu tehnologije, je robot Vatson(eng.~{\em Watson}) za koga je zamisao bila da može da odgovori na sva postavljena pitanja i namena je bila u medicinske svrhe. Međutim, ispostavilo se da robot nije radio ono što su oni zamislili i da je samo mogao da razume reči i da ih čita, i uprkos kritikama od strane stručnih ljudi, IBM odlučuje da Vatson-a izbaci u prodaju. Prodaja nije prošla dobro i to je najveći neuspeh ove kompanije. Odlučuju se da ga prodaju 2022. godine.

Kompanija se 2021. godine deli na dve dela, kao strategija za napredovanje. Odvaja se odeljenje Managed Infrastructure Services, a ostatak kompanije će imati fokus na hibridnu cloud platformu, kao i implementaciju tehnologija koje se tiču veštačke inteligencije.


\begin{table}[h!]
\begin{center}
\caption{Zarada kompanije IBM u prethodnih 5 godina i broj zaposlenih. \cite{tabela}}
\begin{tabular}{|c|c|c|c|c|} \hline
Year& Net income in mil.USD& Total assets in mil. USD& Price per share in USD& Employees \\ \hline
2017	&5,753	&125,356	&149.76	&366,600\\ \hline
2018	&8,723	&123,382	&139.90	&350,600\\ \hline
2019	&9,400	&152,186	&126.85	&352,600\\ \hline
2020	&5,590	&155,971	&125.88	&345,900\\ \hline
2021	&5,743	&132,001	&133.66	&282,100\\ \hline
\end{tabular}
\label{tab:tabela1}
\end{center}
\end{table}

\section{Zaključak}
OVDE CU NAPISATI NEKI ZAKLJUCAK NA KRAJU %TODO

\addcontentsline{toc}{section}{Literatura}
\renewcommand{\refname}{Literatura}
\begin{thebibliography}{9}
\bibliographystyle{unsrt}
\bibitem{lit1} What happend to IBM? 
\url{https://www.thecoldwire.com/what-happened-to-ibm}
\bibitem{tabela} Wikipedia %PROVERITI DA LI JE WIKI IZVORNA LITERATURA
\url{https://en.wikipedia.org/wiki/IBM}
\end{thebibliography}

\end{document}
